\begin{frame}[fragile]

\frametitle{Git: Working with remote}

\begin{block}{Remote}
Remote repositories are versions of your project that are hosted on the Internet or network somewhere. Collaborating with others involves managing these remote repositories and \textbf{pushing and pulling data} to and from them when you need to share work
\end{block}

\pause

Add a remote:
\begin{lstlisting}[language=Bash]
git remote add <remote_name> <remote_url>
\end{lstlisting}

\end{frame}

\begin{frame}[fragile]

\frametitle{Git: Working with remote}

Pushing data:

\begin{lstlisting}[language=Bash]
git push <remote_name> <branch_name>
\end{lstlisting}

\begin{figure}
\includegraphics<1>[scale=0.3]{pushing-1.png}
\includegraphics<2>[scale=0.3]{pushing-2.png}
\caption{Pushing changes}
\label{fig:pushing-changes}
\end{figure}

\end{frame}

\begin{frame}[fragile]

\frametitle{Git: Working with remote}


Pulling data:

\begin{lstlisting}[language=Bash]
git fetch <remote_name> <branch_name>
git pull <remote_name> <branch_name>
\end{lstlisting}

\begin{figure}
\includegraphics<1>[scale=0.3]{pulling-1.png}
\includegraphics<2>[scale=0.3]{pulling-2.png}
\caption{Pulling changes}
\label{fig:pulling-changes}
\end{figure}

\end{frame}