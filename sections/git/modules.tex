\begin{frame}

\frametitle{Git: Modules}

\begin{block}{Mono project metodology}
Mono-project metodology consists of keeping all components in one repository.
\end{block}

\begin{block}{Multi project metodology}
Multi-project metodology consists of creating a project for each component
\end{block}

\end{frame}

\begin{frame}

\frametitle{Git: Modules}

\begin{figure}
\centering
\includegraphics<1>[scale=0.3]{mono-project-metodology.png}
\includegraphics<2>[scale=0.3]{multi-project-metodology.png}
\caption{\only<1>{Mono project metodology example} \only<2>{Multi project metodology example}}
\label{fig:mono-multi-project-metodologies}
\end{figure}


\end{frame}


\begin{frame}

\frametitle{Git: Modules}


\begin{block}{Problem}
In multi project metodology, project dependencies (own or external) are needed by your project. The version management of those, is not straightforward.
\end{block}

\pause

\begin{block}{Solution}
Link the dependent repository to our repository so that you could update the library dependency easily.
\end{block}

\end{frame}

\begin{frame}[fragile]

\frametitle{Git: Modules}

\begin{lstlisting}[language=Bash]
# Relative path to repo
# https://gitlab.ice.csic.es/emiliogq/git-worshop/
# https://gitlab.ice.csic.es/ieec/labs
git submodule add ../ieec/labs

# Will update all submodules to the latest version. 
# A commit on your repo shall be required.
git submodule update --remote
\end{lstlisting}


\end{frame}
